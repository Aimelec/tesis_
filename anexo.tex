\chapter{Anexo}

\large\textbf{Hiperparámetros del modelo}
\vspace{0.6cm}

\begin{table}[H]
    \centering
    \begin{tblr}{|>{\centering\arraybackslash}m{4cm}|>{\centering\arraybackslash}m{7cm}|>{\centering\arraybackslash}m{3cm}|}
        \hline
        \textbf{Hiperparámetro} & \textbf{Descripción} & \textbf{Valor por defecto} \\
        \hline
        layer\_num & Cantidad de celdas LSTM que presenta la red neuronal & 3 \\
        \hline
        embed\_size & Tamaño del embedding resultante de la capa de embeddings & 300 \\
        \hline
        hidden\_size & Dimensión del resultado de la capa oculta de la celda LSTM, lo que se conoce como h\_t & 1150 \\
        \hline
        w\_drop & Weight drop correspondiente al interior de la capa de las celdas LSTM & 0.5 \\
        \hline
        dropout\_i & Dropout que se aplica a los vectores de palabras resultantes de la capa de embeddings & 0.4 \\
        \hline
        dropout\_l & Dropout que se aplica entre medio de las conexiones entre una celda LSTM y otra & 0.3 \\
        \hline
        dropout\_o & Dropout que se aplica al resultado de la última capa LSTM & 0.4 \\
        \hline
        dropout\_e & Dropout que se aplica a la capa de embeddings & 0.1 \\
        \hline
        winit & Valor de inicialización de los pesos de la capa de embeddings & 0.1 \\
        \hline
        batch\_size & Tamaño del batch utilizado para entrenar & 30 \\
        \hline
        valid\_batch\_size & Tamaño del batch utilizado para medir performance en el corpus de validación & 10 \\
        \hline
        bptt & Tamaño de la secuencia* utilizado en el algoritmo de backpropagation through time & 70 \\
        \hline
        ar & Parámetro utilizado en el algoritmo de activation regularization & 2 \\
        \hline
        tar & Parámetro utilizado en el algoritmo de temporary activation regularization & 1 \\
        \hline
    \end{tblr}
    \label{tab:hiperparametros_1}
\end{table}

\begin{table}[H]
    \centering
    \begin{tblr}{|>{\centering\arraybackslash}m{4cm}|>{\centering\arraybackslash}m{7cm}|>{\centering\arraybackslash}m{3cm}|}
        \hline
        weight\_decay & Constante utilizada en la regularización L2 & 1.2e-6 \\
        \hline
        epochs & Cantidad de épocas & 500 \\
        \hline
        lr & Learning rate & 30 \\
        \hline
        max\_grad\_norm & Threshold utilizado en el algoritmo de gradient clipping & 0.25 \\
        \hline
        non\_mono & Tamaño de la ventana que se utiliza para generar el estado de no monotonicidad & 5 \\
        \hline
    \end{tblr}
    \label{tab:hiperparametros_2}
\end{table}

*El tamaño de la secuencia es variable de acuerdo a la época, pero utiliza este valor como base. Para ver mas, consulte en \ref{sec:awd-lstm}
